\documentclass[a4paper,12pt]{report}
\usepackage[utf8]{inputenc}
\usepackage{graphicx}
\usepackage{hyperref}

\begin{document}

% Portada
\begin{titlepage}
    \centering
    {\Huge \bfseries Título del Proyecto \par}
    {\Large Title of the Project \par}
    \vspace{1cm}
    {\Large Autores: \par}
    \vspace{0.5cm}
    {\large Juan Marco Castro Trinidad - A01742821 \par}
    {\large Fedra Fernanda Mandujano López - A00835797 \par}
    {\large Miranda Isabel Rada Chau - A01285243 \par}
    {\large Eliani - A0 \par}
    {\large Alfredo - A0 \par}
    \vfill
    \includegraphics[width=0.4\textwidth]{logo_tec.png} \par
    \vspace{0.5cm}
    {\Large Instituto tecnológico y de estudios superiores de monterrey  \par}
    {\Large Casa Monarca \par}
    \vfill
    {\Large Profesores Responsables del Reto: \par}
    {\large Dr. Otero \par}
    {\large Maestro. Luis Miguel \par}
    \vfill
    {\large \today \par}
\end{titlepage}

% Índice
\tableofcontents
\newpage

% Resumen
\chapter*{Resumen}
\addcontentsline{toc}{chapter}{Resumen}
Aquí va el resumen en español.

\chapter*{Abstract}
\addcontentsline{toc}{chapter}{Abstract}
Here goes the abstract in English.

% Introducción
\chapter{Introducción}
Aquí va la introducción sobre el tema y nicho del objeto de estudio.

% Marco Referencial
\chapter{Marco Referencial}
\section{Marco Teórico}
Aquí va el marco teórico.
\subsection{Aplicaciones de firmas digitales}

Ha habido varios estudios en los que se han comenzado a implementar firmas digitales de este estilo. En Perú, Chaytay (2024) realizó una investigación con la intención de analizar la situación actual de la manera en que se están procesando los documentos importantes que necesitan firmas digitales. Este tipo de firmas son importantes para ver como podría mejorar el manejo administrativo de documentos en las universidades públicas de Perú. A través de este estudio se encontró que el uso de estas firmas trae muchos beneficios para el las entidades pero esto no ha aportado al avance y la agilización de los procesos administrativos de la universidad. (Cachay et al., 2024) Las firmas digitales también se han usado en otros tipos de procesos como las compras electrónicas, el voto electrónico, entre otros. Para que se considere que las firmas digitales y los procesos en los que se utilizan son exitosos es cuando son seguros y tienen factores que aseguran la autenticidad y la integridad de las firmas. Estos aspectos son importantes ya que son los que verifican la identidad de la persona firmando. (Roy & Karforma, 2012)
\section{Marco Contextual}
Aquí va el marco contextual.

\section{Estado del Arte}
Aquí va el estado del arte.

% Introducción
\chapter{Referencias}
Aquí van las referencias.
Cachay Reyes LF, Chang Saldaña JF, Pastor Segura JC, Salirrosas Navarro LS, Castagne Vasquez JY. Document processing system
with digital signatures and administrative management in public universities. A review of the literature. Data and Metadata. 2024; 3:292.
https://doi.org/10.56294/dm2024292
Roy, A., & Karforma, S. (2012). A survey on digital signatures and its applications. Durgapur Society of Management Science, W.B. (INDIA) J. Of Comp. And I.T, 3(1&2), 45–69. https://d1wqtxts1xzle7.cloudfront.net/30180978/J.ofComp._I.T._45-d__%281%2912_%281%29-libre.pdf?1390879655=&response-content-disposition=inline%3B+filename%3DA_survey_on_digital_signatures_and_its_a.pdf&Expires=1741233182&Signature=SrRzcF-gMYrM9s1vlK95dXPEydtdlaOPPBZVrp6HAeJYN4ln6GCQx~jp3A4egeHoY3V~qrXmOB-Zxbnf~IqM87hHERNXRIEECm5syOSglKDFhVIofeR3oHqDFkdimXnb5IpQJjH4~tm-3ZyBHLKQ1L2LCIk8frxjqUgltJUWmndUFtZbUec9KrV0Co5y2YcIC1zFMQbyvRGxiQO3~ho~KbqLzZcJtDuIwDwlawnE7vGGh0jgI-A1uMz3xwyxg7y3qNKNVGrTkTwizGVpV0S5zoVT-HM2SCYYOQYaCKKqUPnZpLbIwjdExWcOWvwayJy6wgOize6s5IEkJIOWzAdGYw__&Key-Pair-Id=APKAJLOHF5GGSLRBV4ZA

‌

\end{document}
