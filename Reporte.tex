\documentclass[a4paper,12pt]{report}
\usepackage[utf8]{inputenc}
\usepackage{graphicx}
\usepackage{hyperref}

\begin{document}

% Portada
\begin{titlepage}
    \centering
    {\Huge \bfseries Título del Proyecto \par}
    {\Large Title of the Project \par}
    \vspace{1cm}
    {\Large Autores: \par}
    \vspace{0.5cm}
    {\large Juan Marco Castro Trinidad - A01742821 \par}
    {\large Fedra - A0 \par}
    {\large Miranda - A0 \par}
    {\large Eliani - A0 \par}
    {\large Alfredo - A0 \par}
    \vfill
    \includegraphics[width=0.4\textwidth]{logo_institucion.png} \par
    \vspace{0.5cm}
    {\Large Instituto tecnológico y de estudios superiores de monterrey  \par}
    {\Large Nombre del Socio Formador \par}
    \vfill
    {\Large Profesores Responsables del Reto: \par}
    {\large Dr. Ucan \par}
    {\large Dr. Salvador \par}
    \vfill
    {\large \today \par}
\end{titlepage}

% Índice
\tableofcontents
\newpage

% Resumen
\chapter*{Resumen}
\addcontentsline{toc}{chapter}{Resumen}
Aquí va el resumen en español.

\chapter*{Abstract}
\addcontentsline{toc}{chapter}{Abstract}
Here goes the abstract in English.

% Introducción
\chapter{Introducción}
Aquí va la introducción sobre el tema y nicho del objeto de estudio.

% Marco Referencial
\chapter{Marco Referencial}
\section{Marco Teórico}
Aquí va el marco teórico.

\section{Marco Contextual}
Aquí va el marco contextual.

\section{Estado del Arte}
Aquí va el estado del arte.

% Introducción
\chapter{Referencias}
Aquí van las referencias.

\end{document}